\documentclass{article}

\usepackage{hcar}

\begin{document}

% Agda-NA.tex
\begin{hcarentry}[section,updated]{Agda}
\label{agda}
\report{Ulf Norell}%11/13
\status{actively developed}
\participants{Ulf Norell, Nils Anders Danielsson, Andreas Abel,
Jesper Cockx, Makoto Takeyama,
Stevan Andjelkovic, Jean-Philippe Bernardy, James Chapman,
Dominique Devriese, P\'eter Divi\'anszky,
Fredrik Nordvall Forsberg, Olle Fredriksson, Daniel Gustafsson,
Alan Jeffrey, Fredrik Lindblad, Guilhem Moulin, Nicolas Pouillard, Andrés Sicard-Ramírez
and many others}
\makeheader

Agda is a dependently typed functional programming language (developed
using Haskell). A central feature of Agda is inductive families,
i.e., GADTs which can be indexed by \emph{values} and not just types.
The language also supports coinductive types, parameterized modules,
and mixfix operators, and comes with an \emph{interactive}
interface---the type checker can assist you in the development of your
code.

A lot of work remains in order for Agda to become a full-fledged
programming language (good libraries, mature compilers, documentation,
etc.), but already in its current state it can provide lots of value as a
platform for research and experiments in dependently typed programming.

A lot has happened in the Agda project and community during the past year.
For instance:
\begin{itemize}
\item Agda~2.5.1 was released in April 2016.
\item An unprecedented amount of Agda documentation is hosted at
\url{http://agda.readthedocs.org/en/stable/} and is being continuously improved.
\item An all new metaprogramming interface (based on David Christiansen's
work in Idris) gives tactic programmers fine-grained control over the
type checking engine.
\item A new unification algorithm for pattern matching has been implemented
and published ({\em Unifiers as equivalences: proof-relevant unification of
dependently typed data, ICFP 2016}) by Jesper Cockx et al.
\end{itemize}
Release of Agda~2.5.2 is planned for late 2016/early 2017.

\FurtherReading
  The Agda Wiki: \url{http://wiki.portal.chalmers.se/agda/}
\end{hcarentry}

\end{document}
