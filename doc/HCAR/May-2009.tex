\documentclass{article}

\usepackage{hcar}

\begin{document}

\begin{hcarentry}[updated]{Agda}
\label{agda}
\report{Nils Anders Danielsson}%05/09
\status{actively developed}
\participants{Ulf Norell and many others}
\makeheader

Do you crave for highly expressive types, but do not want to resort to
type-class hackery? Then Agda might provide a view of what the future
has in store for you.

Agda is a dependently typed functional programming language (developed
using Haskell). The language has inductive families, i.e.\ GADTs which
can be indexed by \emph{values} and not just types. Other goodies
include coinductive types, parameterized modules, mixfix operators,
and an \emph{interactive} Emacs interface (the type checker can assist
you in the development of your code).

A lot of work remains in order for Agda to become a full-fledged
programming language (good libraries, mature compilers, documentation,
etc.), but already in its current state it can provide lots of fun as
a platform for experiments in dependently typed programming.

New since last time:
\begin{itemize}
\item Versions 2.2.0 and 2.2.2 have been released. The previous
  release was in 2007, so the new versions include lots of changes.
\item Agda is now available on Hackage (\texttt{cabal install
    Agda-executable}).
\item Highlighted, hyperlinked HTML can be generated from Agda source
  code using \texttt{agda --html}.
\item The Agda Wiki is better organized, so it should be easier for a
  newcomer to find relevant information.
\end{itemize}

\FurtherReading
  The Agda Wiki: \url{http://wiki.portal.chalmers.se/agda/}
\end{hcarentry}

\end{document}
