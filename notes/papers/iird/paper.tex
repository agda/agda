\documentclass[11pt]{article}

%% ODER: format ==         = "\mathrel{==}"
%% ODER: format /=         = "\neq "
\makeatletter

\usepackage{amstext}
\usepackage{amssymb}
\usepackage{stmaryrd}
\DeclareFontFamily{OT1}{cmtex}{}
\DeclareFontShape{OT1}{cmtex}{m}{n}
  {<5><6><7><8>cmtex8
   <9>cmtex9
   <10><10.95><12><14.4><17.28><20.74><24.88>cmtex10}{}
\DeclareFontShape{OT1}{cmtex}{m}{it}
  {<-> ssub * cmtt/m/it}{}
\newcommand{\texfamily}{\fontfamily{cmtex}\selectfont}
\DeclareFontShape{OT1}{cmtt}{bx}{n}
  {<5><6><7><8>cmtt8
   <9>cmbtt9
   <10><10.95><12><14.4><17.28><20.74><24.88>cmbtt10}{}
\DeclareFontShape{OT1}{cmtex}{bx}{n}
  {<-> ssub * cmtt/bx/n}{}
\newcommand{\tex}[1]{\text{\texfamily#1}}	% NEU

\newcommand{\Sp}{\hskip.33334em\relax}


\newcommand{\Conid}[1]{\mathit{#1}}
\newcommand{\Varid}[1]{\mathit{#1}}
\newcommand{\anonymous}{\kern0.06em \vbox{\hrule\@width.5em}}
\newcommand{\plus}{\mathbin{+\!\!\!+}}
\newcommand{\bind}{\mathbin{>\!\!\!>\mkern-6.7mu=}}
\newcommand{\sequ}{\mathbin{>\!\!\!>}}
\renewcommand{\leq}{\leqslant}
\renewcommand{\geq}{\geqslant}
\usepackage{polytable}

%mathindent has to be defined
\@ifundefined{mathindent}%
  {\newdimen\mathindent\mathindent\leftmargini}%
  {}%

\def\resethooks{%
  \global\let\SaveRestoreHook\empty
  \global\let\ColumnHook\empty}
\newcommand*{\savecolumns}[1][default]%
  {\g@addto@macro\SaveRestoreHook{\savecolumns[#1]}}
\newcommand*{\restorecolumns}[1][default]%
  {\g@addto@macro\SaveRestoreHook{\restorecolumns[#1]}}
\newcommand*{\aligncolumn}[2]%
  {\g@addto@macro\ColumnHook{\column{#1}{#2}}}

\resethooks

\newcommand{\onelinecommentchars}{\quad-{}- }
\newcommand{\commentbeginchars}{\enskip\{-}
\newcommand{\commentendchars}{-\}\enskip}

\newcommand{\visiblecomments}{%
  \let\onelinecomment=\onelinecommentchars
  \let\commentbegin=\commentbeginchars
  \let\commentend=\commentendchars}

\newcommand{\invisiblecomments}{%
  \let\onelinecomment=\empty
  \let\commentbegin=\empty
  \let\commentend=\empty}

\visiblecomments

\newlength{\blanklineskip}
\setlength{\blanklineskip}{1mm}

\newcommand{\hsindent}[1]{\quad}% default is fixed indentation
\newcommand{\NB}{\textbf{NB}}
\newcommand{\Todo}[1]{$\langle$\textbf{To do:}~#1$\rangle$}

\makeatother








\usepackage{ucs}
\usepackage[utf8x]{inputenc}
\usepackage{autofe}
\usepackage{color}

\usepackage{amsthm}
\newtheorem{theorem}{Theorem}[section]
\newtheorem{lemma}[theorem]{Lemma}
\newtheorem{corollary}[theorem]{Corollary}
\newtheorem{definition}[theorem]{Definition}

% Enables greek letters in math environment
\everymath{\SetUnicodeOption{mathletters}}
\everydisplay{\SetUnicodeOption{mathletters}}

% This makes sure that local glyph overrides below are
% chosen.
\DeclareUnicodeOption{localDefs}
\SetUnicodeOption{localDefs}

% For some reason these macros need to be defined.
\newcommand{\textmu}{$\mu$}
\newcommand{\textnu}{$\nu$}

% This character doesn't seem to be defined by ucs.sty.
\DeclareUnicodeCharacter{"21A6}{\ensuremath{\mapsto}}


% Administrative
\newcommand\TODO[1]{\par{
    \setbox0=\vbox{{\small \parbox{3.5cm}{#1}}} %{\parbox{3.5cm}{{\small #1}}}
    \hskip-4.5cm\copy0
    \vskip-\ht0
    \vskip-2mm
}\par
}

% Misc
\newcommand\Or{~~|~~}
\newcommand\C{\mathcal C}
\newcommand\Pair[1]{\langle#1\rangle}

% Syntax
\newcommand\SET{\mathsf{Set}}
\newcommand\EL{\mathsf{El}\,}
\newcommand\PI[2]{(#1:#2)\to{}}
\newcommand\LAM[1]{\lambda #1.{}}
\newcommand\Meta[1]{{?_{#1}}}

\newcommand\Subst[3]{#1[#3/#2]}

% Judgement forms

% Core
\newcommand\IsSigCS[1]{\vdash_{#1}}
\newcommand\IsCtxCS[2]{#2\vdash_{#1}}
\newcommand\IsTypeCS[3]{#2\vdash_{#1}#3}
\newcommand\HasTypeCS[4]{#2\vdash_{#1}#3:#4}
\newcommand\EqualTypeCS[4]{#2\vdash_{#1}#3=#4}
\newcommand\EqualCS[5]{#2\vdash_{#1}#3=#4:#5}

\newcommand\IsCtxC[1]{\IsCtxCS{}{#1}}
\newcommand\IsTypeC[2]{\IsTypeCS{}{#1}{#2}}
\newcommand\HasTypeC[3]{\HasTypeCS{}{#1}{#2}{#3}}
\newcommand\EqualTypeC[3]{\EqualTypeCS{}{#1}{#2}{#3}}
\newcommand\EqualC[4]{\EqualCS{}{#1}{#2}{#3}{#4}}

% Implicit core
\newcommand\WithCnstr[2]{#1\rhd{}#2}
\newcommand\Constr[5]{\Pair{#1,#2,#3,#4,#5}}
\newcommand\Solves{\models}
\newcommand\SigConstr[1]{\left|#1\right|}

\newcommand\IsSigIS[2]{\WithCnstr{#1}{\IsSigCS{#2}}}
\newcommand\IsCtxIS[3]{\WithCnstr{#1}{\IsCtxCS{#2}{#3}}}
\newcommand\IsTypeIS[4]{\WithCnstr{#1}{\IsTypeCS{#2}{#3}{#4}}}
\newcommand\HasTypeIS[5]{\WithCnstr{#1}{\HasTypeCS{#2}{#3}{#4}{#5}}}
\newcommand\EqualTypeIS[5]{\WithCnstr{#1}{\EqualTypeCS{#2}{#3}{#4}{#5}}}
\newcommand\EqualIS[6]{\WithCnstr{#1}{\EqualCS{#2}{#3}{#4}{#5}{#6}}}

\newcommand\IsCtxI[2]{\IsCtxIS{#1}{}{#2}}
\newcommand\IsTypeI[3]{\IsTypeIS{#1}{}{#2}{#3}}
\newcommand\HasTypeI[4]{\HasTypeIS{#1}{}{#2}{#3}{#4}}
\newcommand\EqualTypeI[4]{\EqualTypeIS{#1}{}{#2}{#3}{#4}}
\newcommand\EqualI[5]{\EqualIS{#1}{}{#2}{#3}{#4}{#5}}




\title{Encoding indexed inductive types using the identity type}
\author{Ulf Norell}

\begin{document}
\maketitle
\begin{abstract}
    An indexed inductive-recursive definition (IIRD) simultaneously defines an
    indexed family of sets and a recursive function over this family.  This
    notion is sufficiently powerful to capture essentially all definitions of
    sets in Martin-Löf type theory.

    I show that it is enough to have one particular indexed inductive type,
    namely the intensional identity relation, to be able to interpret all IIRD
    as non-indexed definitions.
    
    The proof is formally verified in Agda.
\end{abstract}

\section{Introduction}

% Describe the current state of affairs

Indexed induction recursion is the thing.

% Indentify gap

Dybjer and Setzer~\cite{dybjer:indexed-ir} show that in an extensional theory
generalised IIRD can be interpreted by restricted IIRD~\cite{dybjer:jsl}. We're
not using an extensional theory though.

\TODO{What's the relation between restricted IIRD and IRD?}

% Fill gap

I improve on this result showing that it is enough to add intensional equality.
The proof is formalised in Agda.

Non-indexed definitions are simpler(?), so if can get away with just adding the
identity type we get simpler meta theory that if we would add indexed
definitions directly.

\section{The Logical Framework}

    Martin-Löf's logical framework~\cite{nordstrom:book} extended with sigma
    types ($\SIGMA x A B$), $\Zero$, $\One$, and $\Two$.

    \TODO{what about $Π$ in Set? Used on the meta level but probably not on the object level.}

    $\HasType {Γ} x A$

    $\IsType {Γ} A$

    $\PI x A B$

    $\SIGMA x A B$

\section{The Identity Type}

There are many versions.

\begingroup\par\noindent\advance\leftskip\mathindent\(
\begin{pboxed}\SaveRestoreHook
\column{B}{@{}l@{}}
\column{5}{@{}l@{}}
\column{11}{@{}l@{}}
\column{E}{@{}l@{}}
\fromto{5}{11}{{}(==){}}
\fromto{11}{E}{{}\mathrel{:}\{\mskip1.5mu \Conid{A}\mathrel{:}\Set\mskip1.5mu\}\to \Conid{A}\to \Conid{A}\to \Set{}}
\nextline
\fromto{5}{11}{{}\mathsf{refl}{}}
\fromto{11}{E}{{}\mathrel{:}\{\mskip1.5mu \Conid{A}\mathrel{:}\Set\mskip1.5mu\}(\Varid{x}\mathrel{:}\Conid{A})\to \Varid{x}==\Varid{x}{}}
\ColumnHook
\end{pboxed}
\)\par\noindent\endgroup\resethooks

Martin-Löf identity relation, introduced in 1973~\cite{martin-lof:predicative}.

\begin{definition}[Martin-Löf elimination]

The Martin-Löf elimination rule (sometimes called $J$) has the type

\begingroup\par\noindent\advance\leftskip\mathindent\(
\begin{pboxed}\SaveRestoreHook
\column{B}{@{}l@{}}
\column{12}{@{}l@{}}
\column{E}{@{}l@{}}
\fromto{B}{12}{{}\Varid{elim\char95 ML}\mathrel{:}{}}
\fromto{12}{E}{{}\{\mskip1.5mu \Conid{A}\mathrel{:}\Set\mskip1.5mu\}(\Conid{C}\mathrel{:}(\Varid{x},\Varid{y}\mathrel{:}\Conid{A})\to \Varid{x}==\Varid{y}\to \Set)\to {}}
\nextline
\fromto{12}{E}{{}((\Varid{x}\mathrel{:}\Conid{A})\to \Conid{C}\;\Varid{x}\;\Varid{x}\;(\mathsf{refl}\;\Varid{x}))\to {}}
\nextline
\fromto{12}{E}{{}(\Varid{x},\Varid{y}\mathrel{:}\Conid{A})\;(\Varid{p}\mathrel{:}\Varid{x}==\Varid{y})\to \Conid{C}\;\Varid{x}\;\Varid{y}\;\Varid{p}{}}
\ColumnHook
\end{pboxed}
\)\par\noindent\endgroup\resethooks

and the corresponding computation rule is
\begingroup\par\noindent\advance\leftskip\mathindent\(
\begin{pboxed}\SaveRestoreHook
\column{B}{@{}l@{}}
\column{3}{@{}l@{}}
\column{E}{@{}l@{}}
\fromto{3}{E}{{}\Varid{elim\char95 ML}\;\Conid{C}\;\Varid{h}\;\Varid{x}\;\Varid{x}\;(\mathsf{refl}\;\Varid{x})\mathrel{=}\Varid{h}\;\Varid{x}{}}
\ColumnHook
\end{pboxed}
\)\par\noindent\endgroup\resethooks
\end{definition}

\begin{definition}[Paulin elimination]
Paulin identity relation~\cite{pfenning-paulin:inductive-coc}.

\begingroup\par\noindent\advance\leftskip\mathindent\(
\begin{pboxed}\SaveRestoreHook
\column{B}{@{}l@{}}
\column{11}{@{}l@{}}
\column{E}{@{}l@{}}
\fromto{B}{11}{{}\Varid{elim\char95 P}\mathrel{:}{}}
\fromto{11}{E}{{}\{\mskip1.5mu \Conid{A}\mathrel{:}\Set\mskip1.5mu\}(\Varid{x}\mathrel{:}\Conid{A})\;(\Conid{C}\mathrel{:}(\Varid{y}\mathrel{:}\Conid{A})\to \Varid{x}==\Varid{y}\to \Set)\to {}}
\nextline
\fromto{11}{E}{{}\Conid{C}\;\Varid{x}\;(\mathsf{refl}\;\Varid{x})\to (\Varid{y}\mathrel{:}\Conid{A})\;(\Varid{p}\mathrel{:}\Varid{x}==\Varid{y})\to \Conid{C}\;\Varid{y}\;\Varid{p}{}}
\ColumnHook
\end{pboxed}
\)\par\noindent\endgroup\resethooks

The corresponding computation rule is
\begingroup\par\noindent\advance\leftskip\mathindent\(
\begin{pboxed}\SaveRestoreHook
\column{B}{@{}l@{}}
\column{3}{@{}l@{}}
\column{E}{@{}l@{}}
\fromto{3}{E}{{}\Varid{elim\char95 P}\;\Varid{x}\;\Conid{C}\;\Varid{h}\;\Varid{x}\;(\mathsf{refl}\;\Varid{x})\mathrel{=}\Varid{h}{}}
\ColumnHook
\end{pboxed}
\)\par\noindent\endgroup\resethooks
\end{definition}

\begin{lemma}
    Martin-Löf elimination can be defined in terms of Paulin elimination.
\end{lemma}

\begin{proof}

Trivial.
\begingroup\par\noindent\advance\leftskip\mathindent\(
\begin{pboxed}\SaveRestoreHook
\column{B}{@{}l@{}}
\column{3}{@{}l@{}}
\column{E}{@{}l@{}}
\fromto{3}{E}{{}\Varid{elim\char95 ML}\;\Conid{C}\;\Varid{h}\;\Varid{x}\;\Varid{y}\;\Varid{p}\mathrel{=}\Varid{elim\char95 P}\;\Varid{x}\;(\lambda \Varid{z}\;\Varid{q}.~\Conid{C}\;\Varid{x}\;\Varid{z}\;\Varid{q})\;(\Varid{h}\;\Varid{x})\;\Varid{y}\;\Varid{p}{}}
\ColumnHook
\end{pboxed}
\)\par\noindent\endgroup\resethooks
\end{proof}

\begin{lemma}
    Paulin elimination can be defined in terms of Martin-Löf elimination.
\end{lemma}

\begin{proof}
    This proof is slightly more involved.

    We first define the substitution rule
\begingroup\par\noindent\advance\leftskip\mathindent\(
\begin{pboxed}\SaveRestoreHook
\column{B}{@{}l@{}}
\column{5}{@{}l@{}}
\column{14}{@{}l@{}}
\column{33}{@{}l@{}}
\column{E}{@{}l@{}}
\fromto{5}{14}{{}\Varid{subst}\mathrel{:}{}}
\fromto{14}{E}{{}\{\mskip1.5mu \Conid{A}\mathrel{:}\Set\mskip1.5mu\}(\Conid{C}\mathrel{:}\Conid{A}\to \Set)\;(\Varid{x},\Varid{y}\mathrel{:}\Conid{A}){}}
\nextline
\fromto{14}{E}{{}\Varid{x}==\Varid{y}\to \Conid{C}\;\Varid{x}\to \Conid{C}\;\Varid{y}{}}
\nextline
\fromto{5}{33}{{}\Varid{subst}\;\Conid{C}\;\Varid{x}\;\Varid{y}\;\Varid{p}\;\Conid{Cx}\mathrel{=}\Varid{elim\char95 ML}\;{}}
\fromto{33}{E}{{}(\lambda \Varid{a}\;\Varid{b}\;\Varid{q}.~\Conid{C}\;\Varid{a}\to \Conid{C}\;\Varid{b})\;{}}
\nextline
\fromto{33}{E}{{}(\lambda \Varid{a}\;\Conid{Ca}.~\Conid{Ca})\;\Varid{x}\;\Varid{y}\;\Varid{p}\;\Conid{Cx}{}}
\ColumnHook
\end{pboxed}
\)\par\noindent\endgroup\resethooks

    Now define

\begingroup\par\noindent\advance\leftskip\mathindent\(
\begin{pboxed}\SaveRestoreHook
\column{B}{@{}l@{}}
\column{5}{@{}l@{}}
\column{E}{@{}l@{}}
\fromto{5}{E}{{}\Conid{E}\mathrel{:}\{\mskip1.5mu \Conid{A}\mathrel{:}\Set\mskip1.5mu\}(\Varid{x}\mathrel{:}\Conid{A})\to \Set{}}
\nextline
\fromto{5}{E}{{}\Conid{E}\;\Varid{x}\mathrel{=}(\Varid{y}\mathrel{:}\Conid{A})×(\Varid{x}==\Varid{y}){}}
\ColumnHook
\end{pboxed}
\)\par\noindent\endgroup\resethooks

    We can prove that any element of \ensuremath{\Conid{E}\;\Varid{x}} is in fact equal to \ensuremath{(\Varid{x},\mathsf{refl}\;\Varid{x})}.

\begingroup\par\noindent\advance\leftskip\mathindent\(
\begin{pboxed}\SaveRestoreHook
\column{B}{@{}l@{}}
\column{5}{@{}l@{}}
\column{E}{@{}l@{}}
\fromto{5}{E}{{}\Varid{uniqE}\mathrel{:}\{\mskip1.5mu \Conid{A}\mathrel{:}\Set\mskip1.5mu\}(\Varid{x},\Varid{y}\mathrel{:}\Conid{A})\;(\Varid{p}\mathrel{:}\Varid{x}==\Varid{y})\to (\Varid{x},\mathsf{refl}\;\Varid{x})==(\Varid{y},\Varid{p}){}}
\nextline
\fromto{5}{E}{{}\Varid{uniqE}\mathrel{=}\Varid{elim}\;(\lambda \Varid{x}\;\Varid{y}\;\Varid{p}.~(\Varid{x},\mathsf{refl}\;\Varid{x})==(\Varid{y},\Varid{p}))\;\mathsf{refl}{}}
\ColumnHook
\end{pboxed}
\)\par\noindent\endgroup\resethooks

\begingroup\par\noindent\advance\leftskip\mathindent\(
\begin{pboxed}\SaveRestoreHook
\column{B}{@{}l@{}}
\column{5}{@{}l@{}}
\column{31}{@{}l@{}}
\column{E}{@{}l@{}}
\fromto{5}{31}{{}\Varid{elim\char95 P}\;\Varid{x}\;\Conid{C}\;\Varid{h}\;\Varid{y}\;\Varid{p}\mathrel{=}\Varid{subst}\;{}}
\fromto{31}{E}{{}(\lambda \Varid{z}.~\Conid{C}\;(π_0\;\Varid{z})\;(π_1\;\Varid{z}))\;{}}
\nextline
\fromto{31}{E}{{}(\Varid{x},\mathsf{refl}\;\Varid{x})\;(\Varid{y},\Varid{p})\;(\Varid{uniqE}\;\Varid{x}\;\Varid{y}\;\Varid{p})\;\Varid{h}{}}
\ColumnHook
\end{pboxed}
\)\par\noindent\endgroup\resethooks

    Note that in an impredicative setting there is a simpler proof due to
    Streicher~\cite{streicher:habilitation}.

\end{proof}

\begin{theorem}
    Martin-Löf elimination and Paulin elimination are equivalent.
\end{theorem}

Streicher axiom K. Not valid~\cite{HofmannM:gromru}. Fortunately we don't need it.

In the following we will use Paulin elimination.

\section{Indexed Induction Recursion}

Very fancy and general types~\cite{dybjer:indexed-ir}.

\TODO{How much details from \cite{dybjer:indexed-ir}?}

Codes for IIRD.

\begingroup\par\noindent\advance\leftskip\mathindent\(
\begin{pboxed}\SaveRestoreHook
\column{B}{@{}l@{}}
\column{5}{@{}l@{}}
\column{9}{@{}l@{}}
\column{16}{@{}l@{}}
\column{E}{@{}l@{}}
\fromto{5}{E}{{}\mathbf{data}\;\Conid{OP}\;(\Conid{I}\mathrel{:}\Set)\;(\Conid{D}\mathrel{:}\Conid{I}\to \Type)\;(\Conid{E}\mathrel{:}\Type)\;\mathbf{where}{}}
\nextline
\fromto{5}{9}{{}\hsindent{4}{}}
\fromto{9}{16}{{}ι{}}
\fromto{16}{E}{{}\mathrel{:}\Conid{E}\to \Conid{OP}\;\Conid{I}\;\Conid{D}\;\Conid{E}{}}
\nextline
\fromto{5}{9}{{}\hsindent{4}{}}
\fromto{9}{16}{{}σ{}}
\fromto{16}{E}{{}\mathrel{:}(\Conid{A}\mathrel{:}\Set)\;(γ\mathrel{:}\Conid{A}\to \Conid{OP}\;\Conid{I}\;\Conid{D}\;\Conid{E})\to \Conid{OP}\;\Conid{I}\;\Conid{D}\;\Conid{E}{}}
\nextline
\fromto{5}{9}{{}\hsindent{4}{}}
\fromto{9}{16}{{}δ{}}
\fromto{16}{E}{{}\mathrel{:}(\Conid{A}\mathrel{:}\Set)\;(\Varid{i}\mathrel{:}\Conid{A}\to \Conid{I})\;(γ\mathrel{:}((\Varid{a}\mathrel{:}\Conid{A})\to \Conid{D}\;(\Varid{i}\;\Varid{a})))\to \Conid{OP}\;\Conid{I}\;\Conid{D}\;\Conid{E}{}}
\ColumnHook
\end{pboxed}
\)\par\noindent\endgroup\resethooks

\subsection{Examples}

Intensional identity relation (Paulin version).

\begingroup\par\noindent\advance\leftskip\mathindent\(
\begin{pboxed}\SaveRestoreHook
\column{B}{@{}l@{}}
\column{5}{@{}l@{}}
\column{E}{@{}l@{}}
\fromto{B}{E}{{}\mathbf{data}\;(==)\{\mskip1.5mu \Conid{A}\mathrel{:}\Set\mskip1.5mu\}(\Varid{x}\mathrel{:}\Conid{A})\mathrel{:}\Conid{A}\to \Set\;\mathbf{where}{}}
\nextline
\fromto{B}{5}{{}\hsindent{5}{}}
\fromto{5}{E}{{}\mathsf{refl}\mathrel{:}\Varid{x}==\Varid{x}{}}
\ColumnHook
\end{pboxed}
\)\par\noindent\endgroup\resethooks

The elimination rule for this type is Paulin elimination.

\section{Encoding}

Just add a proof that the index is the right one.

Example:

\begingroup\par\noindent\advance\leftskip\mathindent\(
\begin{pboxed}\SaveRestoreHook
\column{B}{@{}l@{}}
\column{5}{@{}l@{}}
\column{9}{@{}l@{}}
\column{17}{@{}l@{}}
\column{E}{@{}l@{}}
\fromto{5}{E}{{}\mathbf{data}\;\Conid{Even}\mathrel{:}\Conid{Nat}\to \Set\;\mathbf{where}{}}
\nextline
\fromto{5}{9}{{}\hsindent{4}{}}
\fromto{9}{17}{{}\mathsf{evenZ}{}}
\fromto{17}{E}{{}\mathrel{:}\Conid{Even}\;\mathsf{z}{}}
\nextline
\fromto{5}{9}{{}\hsindent{4}{}}
\fromto{9}{17}{{}\mathsf{evenSS}{}}
\fromto{17}{E}{{}\mathrel{:}\{\mskip1.5mu \Varid{n}\mathrel{:}\Conid{Nat}\mskip1.5mu\}\to \Conid{Even}\;\Varid{n}\to \Conid{Even}\;(\mathsf{s}\;(\mathsf{s}\;\Varid{n})){}}
\ColumnHook
\end{pboxed}
\)\par\noindent\endgroup\resethooks

    The corresponding elimination rule is:

\begingroup\par\noindent\advance\leftskip\mathindent\(
\begin{pboxed}\SaveRestoreHook
\column{B}{@{}l@{}}
\column{5}{@{}l@{}}
\column{9}{@{}l@{}}
\column{E}{@{}l@{}}
\fromto{5}{E}{{}elim_{Even}\mathrel{:}{}}
\nextline
\fromto{5}{9}{{}\hsindent{4}{}}
\fromto{9}{E}{{}(\Conid{C}\mathrel{:}(\Varid{n}\mathrel{:}\Conid{Nat})\to \Conid{Even}\;\Varid{n}\to \Set)\to {}}
\nextline
\fromto{5}{9}{{}\hsindent{4}{}}
\fromto{9}{E}{{}\Conid{C}\;\mathsf{z}\;\mathsf{evenZ}\to {}}
\nextline
\fromto{5}{9}{{}\hsindent{4}{}}
\fromto{9}{E}{{}((\Varid{n}\mathrel{:}\Conid{Nat})\;(\Varid{e}\mathrel{:}\Conid{Even}\;\Varid{n})\to \Conid{C}\;\Varid{n}\;\Varid{e}\to \Conid{C}\;(\mathsf{s}\;(\mathsf{s}\;\Varid{n}))\;(\mathsf{evenSS}\;\Varid{e}))\to {}}
\nextline
\fromto{5}{9}{{}\hsindent{4}{}}
\fromto{9}{E}{{}(\Varid{n}\mathrel{:}\Conid{Nat})\;(\Varid{e}\mathrel{:}\Conid{Even}\;\Varid{n})\to \Conid{C}\;\Varid{n}\;\Varid{e}{}}
\ColumnHook
\end{pboxed}
\)\par\noindent\endgroup\resethooks

As a non-indexed type

\begingroup\par\noindent\advance\leftskip\mathindent\(
\begin{pboxed}\SaveRestoreHook
\column{B}{@{}l@{}}
\column{5}{@{}l@{}}
\column{9}{@{}l@{}}
\column{18}{@{}l@{}}
\column{E}{@{}l@{}}
\fromto{5}{E}{{}\mathbf{data}\;\mathit{Even}^{*}\;(\Varid{n}\mathrel{:}\Conid{Nat})\mathrel{:}\Set\;\mathbf{where}{}}
\nextline
\fromto{5}{9}{{}\hsindent{4}{}}
\fromto{9}{18}{{}\mathsf{evenZ}^{*}{}}
\fromto{18}{E}{{}\mathrel{:}\mathsf{z}==\Varid{n}\to \mathit{Even}^{*}\;\Varid{n}{}}
\nextline
\fromto{5}{9}{{}\hsindent{4}{}}
\fromto{9}{18}{{}\mathsf{evenSS}^{*}{}}
\fromto{18}{E}{{}\mathrel{:}(\Varid{m}\mathrel{:}\Conid{Nat})\to \Conid{Even}\;\Varid{m}\to \mathsf{s}\;(\mathsf{s}\;\Varid{m})==\Varid{n}\to \mathit{Even}^{*}\;\Varid{n}{}}
\nextline[\blanklineskip]
\fromto{5}{E}{{}\mathsf{evenZ}\mathrel{:}\mathit{Even}^{*}\;\mathsf{z}{}}
\nextline
\fromto{5}{E}{{}\mathsf{evenZ}\mathrel{=}\mathsf{evenZ}^{*}\;\mathsf{refl}{}}
\nextline[\blanklineskip]
\fromto{5}{E}{{}\mathsf{evenSS}\mathrel{:}(\Varid{n}\mathrel{:}\Conid{Nat})\to \Conid{Even}\;\Varid{n}\to \Conid{Even}\;(\mathsf{s}\;(\mathsf{s}\;\Varid{n})){}}
\nextline
\fromto{5}{E}{{}\mathsf{evenSS}\;\Varid{n}\;\Varid{e}\mathrel{=}\mathsf{evenSS}^{*}\;\Varid{n}\;\Varid{e}\;\mathsf{refl}{}}
\ColumnHook
\end{pboxed}
\)\par\noindent\endgroup\resethooks

    To define the elimination rule we eliminate the proof that the index is the
    expected one. 

    The elimination rule in deduction style: We have
\begingroup\par\noindent\advance\leftskip\mathindent\(
\begin{pboxed}\SaveRestoreHook
\column{B}{@{}l@{}}
\column{3}{@{}l@{}}
\column{8}{@{}l@{}}
\column{E}{@{}l@{}}
\fromto{3}{8}{{}\Varid{hz}{}}
\fromto{8}{E}{{}\mathrel{:}\Conid{C}\;\Varid{z}\;\mathsf{evenZ}{}}
\nextline
\fromto{3}{8}{{}\Varid{hss}{}}
\fromto{8}{E}{{}\mathrel{:}(\Varid{n}\mathrel{:}\Conid{Nat})\;(\Varid{e}\mathrel{:}\Conid{Even}\;\Varid{n})\to \Conid{C}\;\Varid{n}\;\Varid{e}\to \Conid{C}\;(\Varid{s}\;(\Varid{s}\;\Varid{n}))\;(\mathsf{evenSS}\;\Varid{n}\;\Varid{e}){}}
\nextline
\fromto{3}{8}{{}\Varid{n}{}}
\fromto{8}{E}{{}\mathrel{:}\Conid{Nat}{}}
\nextline
\fromto{3}{8}{{}\Varid{e}{}}
\fromto{8}{E}{{}\mathrel{:}\Conid{Even}\;\Varid{n}{}}
\ColumnHook
\end{pboxed}
\)\par\noindent\endgroup\resethooks
    and the goal is
\begingroup\par\noindent\advance\leftskip\mathindent\(
\begin{pboxed}\SaveRestoreHook
\column{B}{@{}l@{}}
\column{3}{@{}l@{}}
\column{E}{@{}l@{}}
\fromto{3}{E}{{}\mathbin{?}\mathrel{:}\Conid{C}\;\Varid{n}\;\Varid{e}{}}
\ColumnHook
\end{pboxed}
\)\par\noindent\endgroup\resethooks
    By eliminating \ensuremath{\Varid{e}} (using the non-indexed elimination) we get two cases

\begingroup\par\noindent\advance\leftskip\mathindent\(
\begin{pboxed}\SaveRestoreHook
\column{B}{@{}l@{}}
\column{3}{@{}l@{}}
\column{7}{@{}l@{}}
\column{E}{@{}l@{}}
\fromto{3}{7}{{}\Varid{p}_{1}{}}
\fromto{7}{E}{{}\mathrel{:}\mathsf{z}==\Varid{n}{}}
\nextline
\fromto{3}{7}{{}\mathbin{?}{}}
\fromto{7}{E}{{}\mathrel{:}\Conid{C}\;\Varid{n}\;(\mathsf{evenZ}^{*}\;\Varid{p}_{1}){}}
\ColumnHook
\end{pboxed}
\)\par\noindent\endgroup\resethooks
    Now we can eliminate \ensuremath{\Varid{p}_{1}}, effectively substituting \ensuremath{\Varid{z}} for \ensuremath{\Varid{n}} and \ensuremath{\mathsf{refl}}
    for \ensuremath{\Varid{p}_{1}} in the goal to obtain the new goal
\begingroup\par\noindent\advance\leftskip\mathindent\(
\begin{pboxed}\SaveRestoreHook
\column{B}{@{}l@{}}
\column{3}{@{}l@{}}
\column{E}{@{}l@{}}
\fromto{3}{E}{{}\mathbin{?}\mathrel{:}\Conid{C}\;\mathsf{z}\;(\mathsf{evenZ}^{*}\;\mathsf{refl}){}}
\ColumnHook
\end{pboxed}
\)\par\noindent\endgroup\resethooks
    This is exactly the type of \ensuremath{\Varid{hz}}.

    In the second case we get
\begingroup\par\noindent\advance\leftskip\mathindent\(
\begin{pboxed}\SaveRestoreHook
\column{B}{@{}l@{}}
\column{3}{@{}l@{}}
\column{7}{@{}l@{}}
\column{E}{@{}l@{}}
\fromto{3}{7}{{}\Varid{ih}{}}
\fromto{7}{E}{{}\mathrel{:}\Conid{C}\;\Varid{m}\;\Varid{e'}{}}
\nextline
\fromto{3}{7}{{}\Varid{p}_{2}{}}
\fromto{7}{E}{{}\mathrel{:}\mathsf{s}\;(\mathsf{s}\;\Varid{m})==\Varid{n}{}}
\nextline
\fromto{3}{7}{{}\mathbin{?}{}}
\fromto{7}{E}{{}\mathrel{:}\Conid{C}\;\Varid{n}\;(\mathsf{evenSS}^{*}\;\Varid{m}\;\Varid{e'}\;\Varid{p}_{2}){}}
\ColumnHook
\end{pboxed}
\)\par\noindent\endgroup\resethooks
    Eliminating \ensuremath{\Varid{p}_{2}} yields
\begingroup\par\noindent\advance\leftskip\mathindent\(
\begin{pboxed}\SaveRestoreHook
\column{B}{@{}l@{}}
\column{3}{@{}l@{}}
\column{E}{@{}l@{}}
\fromto{3}{E}{{}\mathbin{?}\mathrel{:}\Conid{C}\;(\mathsf{s}\;(\mathsf{s}\;\Varid{m}))\;(\mathsf{evenSS}^{*}\;\Varid{m}\;\Varid{e'}\;\mathsf{refl}){}}
\ColumnHook
\end{pboxed}
\)\par\noindent\endgroup\resethooks
    which is the type of
\begingroup\par\noindent\advance\leftskip\mathindent\(
\begin{pboxed}\SaveRestoreHook
\column{B}{@{}l@{}}
\column{3}{@{}l@{}}
\column{E}{@{}l@{}}
\fromto{3}{E}{{}\Varid{hss}\;\Varid{m}\;\Varid{e'}\;\Varid{ih}{}}
\ColumnHook
\end{pboxed}
\)\par\noindent\endgroup\resethooks
Here is the elimination spelled out. To improve readability I present the
elimination of the interpretation using pattern matching notation and explicit
recursion.

\begingroup\par\noindent\advance\leftskip\mathindent\(
\begin{pboxed}\SaveRestoreHook
\column{B}{@{}l@{}}
\column{5}{@{}l@{}}
\column{9}{@{}l@{}}
\column{17}{@{}l@{}}
\column{38}{@{}c@{}}
\column{38E}{@{}l@{}}
\column{43}{@{}c@{}}
\column{43E}{@{}l@{}}
\column{E}{@{}l@{}}
\fromto{5}{38}{{}elim_{Even}\;\Conid{C}\;\Varid{hz}\;\Varid{hss}\;\Varid{n}\;(\mathsf{evenZ}^{*}\;\Varid{p}){}}
\fromto{38}{38E}{{}\mathrel{=}{}}
\nextline
\fromto{5}{9}{{}\hsindent{4}{}}
\fromto{9}{E}{{}\Varid{elim\char95 P}\;\mathsf{z}\;(\lambda \Varid{z}\;\Varid{q}.~\Conid{C}\;\mathsf{z}\;\mathsf{evenZ}^{*})\;\Varid{hz}\;\Varid{n}\;\Varid{p}{}}
\nextline
\fromto{5}{43}{{}elim_{Even}\;\Conid{C}\;\Varid{hz}\;\Varid{hss}\;\Varid{n}\;(\mathsf{evenSS}^{*}\;\Varid{m}\;\Varid{e}\;\Varid{p}){}}
\fromto{43}{43E}{{}\mathrel{=}{}}
\nextline
\fromto{5}{9}{{}\hsindent{4}{}}
\fromto{9}{17}{{}\Varid{elim\char95 P}\;{}}
\fromto{17}{E}{{}(\mathsf{s}\;(\mathsf{s}\;\Varid{m}))\;{}}
\nextline
\fromto{17}{E}{{}(\lambda \Varid{z}\;\Varid{q}.~\Conid{C}\;(\mathsf{s}\;(\mathsf{s}\;\Varid{m}))\;(\mathsf{evenSS}^{*}\;\Varid{e}))\;{}}
\nextline
\fromto{17}{E}{{}(\Varid{hss}\;\Varid{m}\;\Varid{e}\;(elim_{Even}\;\Conid{C}\;\Varid{hz}\;\Varid{hss}\;\Varid{m}\;\Varid{e}))\;{}}
\nextline
\fromto{17}{E}{{}\Varid{n}\;\Varid{p}{}}
\ColumnHook
\end{pboxed}
\)\par\noindent\endgroup\resethooks

This gives us the correct type for the elimination rule, but we also need the
correct computation rules. Namely
\begingroup\par\noindent\advance\leftskip\mathindent\(
\begin{pboxed}\SaveRestoreHook
\column{B}{@{}l@{}}
\column{3}{@{}l@{}}
\column{5}{@{}l@{}}
\column{37}{@{}l@{}}
\column{51}{@{}l@{}}
\column{E}{@{}l@{}}
\fromto{3}{37}{{}elim_{Even}\;\Conid{C}\;\Varid{hz}\;\Varid{hss}\;\mathsf{z}\;{}}
\fromto{37}{51}{{}\mathsf{evenZ}{}}
\fromto{51}{E}{{}\mathrel{=}\Varid{hz}{}}
\nextline
\fromto{3}{37}{{}elim_{Even}\;\Conid{C}\;\Varid{hz}\;\Varid{hss}\;(\mathsf{s}\;(\mathsf{s}\;\Varid{n}))\;{}}
\fromto{37}{51}{{}(\mathsf{evenSS}\;\Varid{n}\;\Varid{e}){}}
\fromto{51}{E}{{}\mathrel{=}{}}
\nextline
\fromto{3}{5}{{}\hsindent{2}{}}
\fromto{5}{E}{{}\Varid{hss}\;\Varid{n}\;\Varid{e}\;(elim_{Even}\;\Conid{C}\;\Varid{hz}\;\Varid{hss}\;\Varid{n}\;\Varid{e}){}}
\ColumnHook
\end{pboxed}
\)\par\noindent\endgroup\resethooks
The crucial point here is that the equations talk about \ensuremath{\mathsf{evenZ}} and \ensuremath{\mathsf{evenSS}\;\Varid{n}\;\Varid{e}} and not arbitrary elements of \ensuremath{\mathit{Even}^{*}\;\Varid{n}}. In these particular cases the proof
object is \ensuremath{\mathsf{refl}} so we can use the computation rule of equality elimination to
show the desired computation rules.

\subsection{Proof}

Simple induction over the code for the indexed type.

\section{Related Work}

Peter and Anton obviously~\cite{dybjer:indexed-ir}.

\section{Conclusions}

This is good stuff.

\bibliographystyle{abbrv}
\bibliography{../../../../bib/pmgrefs}

\end{document}