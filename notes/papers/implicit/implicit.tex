% Implicit arguments
% Author: Catarina Coquand and Ulf Norell

%\documentclass[12pt,a4paper]{amsart}
\documentclass[11pt]{article}

\usepackage{amsthm}
\newtheorem{theorem}{Theorem}[section]
\newtheorem{lemma}[theorem]{Lemma}
\newtheorem{corollary}[theorem]{Corollary}
\newtheorem{definition}[theorem]{Definition}


\usepackage{epsf}
\usepackage{epsfig}
%\usepackage{isolatin1}
%\usepackage{a4wide}
\usepackage{verbatim}
\usepackage{proof}
\usepackage{latexsym}
\usepackage{amssymb,amsmath}
\usepackage{stmaryrd}


% Administrative
\newcommand\TODO[1]{\par{
    \setbox0=\vbox{{\small \parbox{3.5cm}{#1}}} %{\parbox{3.5cm}{{\small #1}}}
    \hskip-4.5cm\copy0
    \vskip-\ht0
    \vskip-2mm
}\par
}

% Misc
\newcommand\Or{~~|~~}
\newcommand\C{\mathcal C}
\newcommand\Pair[1]{\langle#1\rangle}

% Syntax
\newcommand\SET{\mathsf{Set}}
\newcommand\EL{\mathsf{El}\,}
\newcommand\PI[2]{(#1:#2)\to{}}
\newcommand\LAM[1]{\lambda #1.{}}
\newcommand\Meta[1]{{?_{#1}}}

\newcommand\Subst[3]{#1[#3/#2]}

% Judgement forms

% Core
\newcommand\IsSigCS[1]{\vdash_{#1}}
\newcommand\IsCtxCS[2]{#2\vdash_{#1}}
\newcommand\IsTypeCS[3]{#2\vdash_{#1}#3}
\newcommand\HasTypeCS[4]{#2\vdash_{#1}#3:#4}
\newcommand\EqualTypeCS[4]{#2\vdash_{#1}#3=#4}
\newcommand\EqualCS[5]{#2\vdash_{#1}#3=#4:#5}

\newcommand\IsCtxC[1]{\IsCtxCS{}{#1}}
\newcommand\IsTypeC[2]{\IsTypeCS{}{#1}{#2}}
\newcommand\HasTypeC[3]{\HasTypeCS{}{#1}{#2}{#3}}
\newcommand\EqualTypeC[3]{\EqualTypeCS{}{#1}{#2}{#3}}
\newcommand\EqualC[4]{\EqualCS{}{#1}{#2}{#3}{#4}}

% Implicit core
\newcommand\WithCnstr[2]{#1\rhd{}#2}
\newcommand\Constr[5]{\Pair{#1,#2,#3,#4,#5}}
\newcommand\Solves{\models}
\newcommand\SigConstr[1]{\left|#1\right|}

\newcommand\IsSigIS[2]{\WithCnstr{#1}{\IsSigCS{#2}}}
\newcommand\IsCtxIS[3]{\WithCnstr{#1}{\IsCtxCS{#2}{#3}}}
\newcommand\IsTypeIS[4]{\WithCnstr{#1}{\IsTypeCS{#2}{#3}{#4}}}
\newcommand\HasTypeIS[5]{\WithCnstr{#1}{\HasTypeCS{#2}{#3}{#4}{#5}}}
\newcommand\EqualTypeIS[5]{\WithCnstr{#1}{\EqualTypeCS{#2}{#3}{#4}{#5}}}
\newcommand\EqualIS[6]{\WithCnstr{#1}{\EqualCS{#2}{#3}{#4}{#5}{#6}}}

\newcommand\IsCtxI[2]{\IsCtxIS{#1}{}{#2}}
\newcommand\IsTypeI[3]{\IsTypeIS{#1}{}{#2}{#3}}
\newcommand\HasTypeI[4]{\HasTypeIS{#1}{}{#2}{#3}{#4}}
\newcommand\EqualTypeI[4]{\EqualTypeIS{#1}{}{#2}{#3}{#4}}
\newcommand\EqualI[5]{\EqualIS{#1}{}{#2}{#3}{#4}{#5}}




\title{Implicit Arguments}
\author{Catarina Coquand and Ulf Norell}
\date{\today}

\begin{document}


\maketitle

\section{Core}

\[\begin{array}{lclr}
    A	   & ::= & \SET \Or \EL M \Or \PI xAA		& types \\
    M	   & ::= & x \Or c \Or f \Or M\,M \Or \LAM xM	& terms \\
    \Gamma & ::= & () \Or \Gamma,x:A			& contexts \\
    \Sigma & ::= & () \Or \Sigma,c:A \Or \Sigma,f:A=M	& signatures \\
\end{array}\]

 There are six kinds of judgement 
\[\begin{array}{ll}
    \IsSigCS\Sigma & \mbox{$\Sigma$ is a valid signature} \\
    \IsCtxCS\Sigma\Gamma & \mbox{$\Gamma$ is a valid context} \\
    \IsTypeCS\Sigma\Gamma A & \mbox{$A$ is a valid type in $\Gamma$} \\
    \HasTypeCS\Sigma\Gamma MA & \mbox{$M$ has type $A$ in $\Gamma$} \\
    \EqualTypeCS\Sigma\Gamma AB & \mbox{$A$ and $B$ are convertible types in $\Gamma$}\\
    \EqualCS\Sigma\Gamma MNA & \mbox{$M$ and $N$ are convertible terms of type $A$ in $\Gamma$} \\
\end{array}\]

 We will leave $\Sigma$ implicit whenever possible.

 The typing rules are as follows:

\TODO{where should we put $\IsSigCS\Sigma$?}

\medskip

 {\em rules for contexts}

\[\begin{array}{ccc}
    \infer{\IsCtxC{}}{}
&&  \infer{\IsCtxC{\Gamma,x:A}}{
      \IsCtxC\Gamma
    & \IsTypeC\Gamma A
    }
\end{array}\]

 {\em rules for types}

\[\begin{array}{ccccc}
    \infer{\IsTypeC\Gamma\SET}{\IsCtxC\Gamma}
&&  \infer{\IsTypeC\Gamma{\EL M}}{\HasTypeC\Gamma M\SET}
&&  \infer{\IsTypeC\Gamma{\PI xAB}}{\IsTypeC{\Gamma,x:A}B}
\end{array}\]

 {\em rules for terms}

\[\begin{array}{ccccc}
    \infer{\HasTypeC\Gamma xA}{
      \IsCtxC\Gamma
    & x:A\in\Gamma
    }
&&  \infer{\HasTypeCS\Sigma\Gamma cA}{
    & \IsCtxC\Gamma
    & c:A\in\Sigma
    }
&&  \infer{\HasTypeCS\Sigma\Gamma fA}{
      \IsCtxC\Gamma
    & f:A=M\in\Sigma
    }
\\{}\\
\multicolumn5c{\begin{array}{ccc}
    \infer{\HasTypeC\Gamma{\LAM xM}{\PI xAB}}{
      \HasTypeC{\Gamma,x:A}MB
    }
&&  \infer{\HasTypeC\Gamma{M\,N}{\Subst BxN}}{
      \HasTypeC\Gamma M{\PI xAB}
    & \HasTypeC\Gamma NA
    }
\end{array}}
\end{array}\]

 {\em conversion rule}

\[
    \infer{\HasTypeC\Gamma MB}{
      \HasTypeC\Gamma MA
    & \EqualTypeC\Gamma AB
    }
\]

% \medskip
% \medskip
% 
% 
%  The general conversion rules are
% 
% $$
% \frac{\Gamma\vdash A}{\Gamma\vdash A=A}~~~~~~
% \frac{\Gamma\vdash A=B}{\Gamma\vdash B=A}~~~~~~
% \frac{\Gamma\vdash A=B~~~\Gamma\vdash B=C}{\Gamma\vdash A=C}
% $$
% 
% $$
% \frac{\Gamma\vdash M:A}{\Gamma\vdash M=M:A}~~~~~
% \frac{\Gamma\vdash M=N:A}{\Gamma\vdash N=M:A}~~~~
% \frac{\Gamma\vdash M=N:A~~~~\Gamma\vdash N=P:A}{\Gamma\vdash M=P:A}
% $$
% 
% $$
% \frac{\Gamma\vdash M=N:A~~~~~\Gamma\vdash A = B}
%      {\Gamma\vdash M=N:B}
% $$
% 
% $$
% \frac{\Gamma,x{:}A\vdash B~~~~~\Gamma\vdash M_1=M_2:A}
%      {\Gamma\vdash B(x=M_1) = B(x=M_2)}
% $$
% 
% \medskip
% 
% The conversion rules for type theory are
% 
% $$
% \frac{}{\Gamma\vdash\SET=\SET}~~~~
% \frac{\Gamma\vdash M_1=M_2:\SET}{\Gamma\vdash\EL~M_1=\EL~M_2}~~~~~~
% \frac{\Gamma\vdash A_1=A_2~~~~~\Gamma,x{:}A_1\vdash B_1=B_2}{\Gamma\vdash (x{:}A_1)\rightarrow B_1 = (x{:}A_2)\rightarrow B_2}
% $$
% 
% $$
% \frac{\Gamma,x{:}A\vdash M_1=M_2:B}{\Gamma\vdash \lambda x.M_1=\lambda x.M_2:(x{:}A)\rightarrow B}
% $$
% 
% $$
% \frac{\Gamma,x{:}A\vdash B~~~~\Gamma\vdash N_1=N_2:(x{:}A)\rightarrow B~~~~\Gamma\vdash M_1=M_2:A}
%      {\Gamma\vdash N_1~M_1 = N_2~M_2:B(x=M_1)}
% $$
% 
% $$
% \frac{\Gamma,x{:}A\vdash N:B~~~~~\Gamma\vdash M:A}
%      {\Gamma\vdash (\lambda x.N)~M = N(x=M): B(x=M)}~~~~~~~
% \frac{\Gamma\vdash A~~~~\Gamma\vdash M:(x{:}A)\rightarrow B}
%      {\Gamma\vdash M = \lambda x.M~x : (x{:}A)\rightarrow B}
% $$
% 
% $$
% \frac{f : A = M \in \Sigma}
%      {\Gamma\vdash f = M : A}
% $$
% 
%  In this presentation of rules, we consider $\lambda$ terms up to  $\alpha$-conversion.
% 
% \subsection{Type checking}
% 
% $$
% \frac{(x{:}A)~\in~\Gamma}{\Gamma\vdash x\downarrow A}~~~~~~~~~
% \frac{\Gamma\vdash M\downarrow (x{:}A)\rightarrow B~~~~\Gamma\vdash N\uparrow A}
%      {\Gamma\vdash M~N\downarrow B(x=N)}
% $$
% and
% $$
% \frac{\Gamma,x{:}A\vdash M\uparrow B}{\Gamma\vdash \lambda x.M\uparrow (x{:}A)\rightarrow B}~~~~
% \frac{\Gamma\vdash M\downarrow A~~~~\Gamma\vdash A = B}{\Gamma\vdash M\uparrow B}
% $$
% 
% $$
% \frac{}{\Gamma\vdash \SET\downarrow}~~~~
% \frac{\Gamma\vdash M\downarrow\SET}{\Gamma\vdash \EL~M\downarrow}~~~~~~
% \frac{\Gamma\vdash A\downarrow~~~~~\Gamma,x{:}A\vdash B\downarrow}{\Gamma\vdash (x{:}A)\rightarrow B\downarrow}
% $$
% 
% \begin{theorem}
% If $M$ is in $\beta$-normal form and $\Gamma\vdash M{:}A$ then $\Gamma\vdash M\uparrow A$.
% \end{theorem}


\section{Implicit Core}

\begin{thebibliography}{9}

\bibitem{Type}
B. Nordstr\"om, K. Petersson, J. Smith.
\newblock{{\em Programming in Martin-L\"of's Type Theory.}}
\newblock{Oxford Science Publications, 1990.}



\end{thebibliography}

\end{document}

